\section{The electrostatic framework}
\label{section: electrostatic framework}

The electrostatic framework considers the screening response of the superlattice perturbed by an externally-sourced potential, $\phi_{ext}$.  The perturbing field induces a screening charge density $\rho_{ind}$, which is governed by the electronic density response function $\chi$,\footnote{I.e., we are ignoring ionic contributions to the electrostatic screening response.  This neglected response would nevertheless contribute to charge density probes like EELS.}
\begin{equation}
    \label{real space chi def}
    \rho_{ind}(r,t) = e^2\int d^3 r' dt' \chi(r,r';t-t')\phi_{ext}(r',t')
    \,\,\,.
\end{equation}
Through Gauss's law, the induced charge density sources the screening field $\phi_{ind}$; in Fourier space,\footnote{I'll be working in S.I. units throughout.}
\begin{equation}
    \label{Fourier induced Gauss law}
    \phi_{ind}(q,\omega) = \frac{\rho_{ind}(q,\omega)}{\e_0 q^2}
    \,\,\,.
\end{equation}


If the superlattice had continuous translation symmetry, the linear response for $\rho_{ind}$ in \eqref{real space chi def} could be expressed in terms of the product of Fourier components for $\chi$ and $\phi_{ext}$.  The superlattice geometry of Fig. \ref{figure: superlattice geometry}, however, has only discrete translation symmetry.  As a result, a potential perturbation with fixed out-of-place wavevector induces a charge density modulation at all reciprocal lattice vectors.  In particular, we have that
\begin{equation}
    \label{superlattice density response structure}
    \chi(q_\parallel,\omega;z,z') = \sum_G e^{iG z} \int \frac{dQ}{2\pi} e^{ik(z-z')}\chi(q_\parallel,\omega;Q+G,Q)
    \,\,\,,
\end{equation}
where $q_\parallel$ is the planar component of the Fourier wavevector and the sum is over all out-of-plane reciprocal vectors.  As is standard, the continuous out-of-plane wavevector---$Q$ in \eqref{superlattice density response structure}---can be restricted to the first Brillouin zone,
\begin{equation}
    \chi(q_\parallel,\omega;z,z') = \sum_{G,G'} e^{iG z} e^{-i G' z'} \int_\text{BZ} \frac{dQ}{2\pi} e^{iQ(z-z')}\chi(q_\parallel,\omega;Q+G,Q+G')
    \,\,\,.
\end{equation}
The best we can do to simplify the linear response relationship in \eqref{real space chi def} is to obtain a matrix equation in the reciprocal lattice vectors of the superlattice,
\begin{equation}
     \rho_{ind}(q_\parallel,\omega;Q+G) = e^2\sum_{G'}\chi(q_\parallel,\omega;Q+G, Q+G')
     \phi_{ext}(q_\parallel,\omega;Q+G')
     \,\,\,.
\end{equation}

The superlattice structure contributes to the screening response through \eqref{Fourier induced Gauss law},
\begin{equation}
     \phi_{ind}(q_\parallel,\omega;Q+G) = \sum_{G'}V(q_\parallel,Q+G)\chi(q_\parallel,\omega;Q+G, Q+G')
     \phi_{ext}(q_\parallel,\omega;Q+G')
     \,\,\,,
\end{equation}
where I've defined the Coulomb interaction,
\begin{equation}
    V(q_\parallel,Q+G):=
    \frac{e^2}{\e_0\lb q_\parallel^2+\lp Q+G\rp^2\rb}
    \,\,\,.
\end{equation}
The {\it total} potential $\phi$ is then made up of three components,
\begin{equation}
    \phi = \phi_0 + \phi_{ind} + \phi_{ext},
\end{equation}
where $\phi_0$ describes the micro-structure of the unperturbed superlattice potential.\footnote{The macroscopic constituents in Fig. \ref{figure: superlattice geometry} are presumed charge-neutral on the whole, but nevertheless contain ionic and electronic charge densities that vary on microscopic length scales.}  As a result, the total potential $\phi$, the electronic density response function $\chi$, and the perturbing potential $\phi_{ext}$ are related according to
\begin{eqnarray}
    \label{microscopic total potential expression}
    \nonumber 
    \phi(q_\parallel,\omega;Q+G) 
    &=&
    \phi_0(q_\parallel, \omega; Q+G) 
    \\ \nonumber &\,& +
    \Bigg(\sum_{G'}
    \lb
    \delta_{G,G'}
    +
    V(q_\parallel,Q+G)
    \chi(q_\parallel,\omega;Q+G, Q+G')
    \rb
    \\ &\,&
    \times
    \phi_{ext}(q_\parallel,\omega;Q+ G')\Bigg)
    \,\,\,.
\end{eqnarray}

The electrostatic relationship in \eqref{microscopic total potential expression} can be greatly simplified if we restrict our study to the long wavelength limit.  We can neglect the micro-structure $\phi_0$ by two separate arguments.  If we strict ourselves to truly dynamic $\omega>0$ response, then $\phi_0$ should not contribute as it describes the static superlattice.  Additionally, $\phi_0$ should not contribute in the long wavelength $G=0$ limit due to charge neutrality.\footnote{This is more subtle, as I need to argue that we don't have any interesting commensurate structures arising from the inter-play of superlattice and microscopic lattice scales.  A truly microscopic analysis might find interesting long wavelength contributions due to these effects.}  Either way, we can safely neglect $\phi_0$ in the subsequent electrostatic analysis by focusing on the $G=0$ response.  On the right-hand side of \eqref{microscopic total potential expression}, the experimenter can control the nature of the probe modeled by $\phi_{ext}$.  In particular, we can consider the superlattice response to a long wavelength perturbation, where $G'=0$.  As a result, the long wavelength response of the superlattice to a long wavelength perturbation is governed by a {\it scalar} equation,
\begin{equation}
    \label{long wavelength total potential}
    \phi(q_\parallel, Q, \omega) 
    =
    \lb
    1
    +
    V(q_\parallel, Q)
    \chi(q_\parallel, Q, \omega)
    \rb
    \phi_{ext}(q_\parallel, Q, \omega)
    \,\,\,,
\end{equation}
where $Q$ remains restricted to the first Brillouin zone.  If we can solve for $\phi$ in the presence of a generic perturbation $\phi_{ext}$, then \eqref{long wavelength total potential} allows us to determine the density response function $\chi$ \eqref{Fourier density response def}.