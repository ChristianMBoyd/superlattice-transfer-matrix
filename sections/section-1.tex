\section{The electrostatic framework}
\label{section: electrostatic framework}

The electrostatic framework considers the screening response of the superlattice perturbed by an externally-sourced potential, $\phi_{ext}$.  The perturbing field induces a screening charge density $\rho_{ind}$, which is governed by the electronic density response function $\chi$,\footnote{I.e., we are ignoring ionic contributions to the electrostatic screening response.  This neglected response would nevertheless contribute to charge density probes like EELS.}
\begin{equation}
    \label{real space chi def}
    \rho_{ind}(r,t) = e^2\int d^3 r' dt' \chi(r,r';t-t')\phi_{ext}(r',t')
    \,\,\,.
\end{equation}
By Gauss's law, the superlattice contribues a screening field $\phi_{ind}$; in Fourier space,
\begin{equation}
    \label{Fourier phi ind}
    \phi_{ind}(q,\omega) = \frac{\rho_{ind}(q,\omega)}{\e_0 q^2}
    \,\,\,.
\end{equation}
At the level of linear response, the (total) potential $\phi$ results from the sum
\begin{equation}
    \label{phi def}
    \phi(q,\omega) = \phi_{ind}(q,\omega) + \phi_{ext}(q,\omega)
    \,\,\,.
\end{equation}


If the superlattice had continuous translation symmetry, the linear response for $\rho_{ind}$ in \eqref{real space chi def} could be expressed in terms of the product of Fourier components for $\chi$ and $\phi_{ext}$.  The superlattice geometry of Fig. \ref{figure: superlattice geometry}, however, has only discrete translation symmetry.  As a result, a potential perturbation with fixed out-of-place wavevector induces a charge density modulation at all reciprocal lattice vectors.  In particular, we have that
\begin{equation}
    \label{Fourier chi expression}
    \chi(q_\parallel,\omega;z,z') = \sum_G e^{iG z} \int \frac{dk}{2\pi} e^{ik(z-z')}\chi(q_\parallel,\omega;k+G,k)
    \,\,\,,
\end{equation}
where $q_\parallel$ is the planar component of the Fourier wavevector and the sum is over all out-of-plane reciprocal vectors.  As is standard, the continuous out-of-plane wavevector $k$ in \eqref{Fourier chi expression} can be restricted to the first Brillouin zone (denoted as $\bar k$ in what follows),
\begin{equation}
    \label{Fourier chi G matrix}
    \chi(q_\parallel,\omega;z,z') = \sum_{G,G'} e^{iG z} e^{-i G' z'} \int_\text{BZ} \frac{d\bar k}{2\pi} e^{i\bar k(z-z')}\chi(q_\parallel,\omega;\bar k+G,\bar k+G')
    \,\,\,.
\end{equation}
The best we can do to simplify the linear response relationship in \eqref{real space chi def} is to obtain a matrix equation in the reciprocal lattice vectors of the superlattice,
\begin{equation}
    \label{planar linear response}
     \rho_{ind}(q_\parallel,\omega;\bar k+G) = e^2\sum_{G'}\chi(q_\parallel,\omega;\bar k+G, \bar k+G')
     \phi_{ext}(q_\parallel,\omega;\bar k+G')
     \,\,\,.
\end{equation}


The expression for the total potential $\phi$ \eqref{phi def} can be understood by first expanding the screening field $\phi_{ind}$ \eqref{Fourier phi ind} in terms of the superlattice contributions,
\begin{equation}
    \label{planar phi ind}
     \phi_{ind}(q_\parallel,\omega;\bar k+G) = \frac{e^2}{\e_0\lb q_\parallel^2+\lp\bar k+G\rp^2\rb} \sum_{G'}\chi(q_\parallel,\omega;\bar k+G, \bar k+G')
     \phi_{ext}(q_\parallel,\omega;\bar k+G')
     \,\,\,.
\end{equation}
Finally, the relationship between the total potential $\phi$, the electronic density response function $\chi$, and the perturbing potential $\phi_{ext}$ can be expressed as
\begin{equation}
    \label{planar phi to phi ext relation}
    \phi(q_\parallel,\omega;\bar k+G) 
    =
    \sum_{G'}
    \lb
    \delta_{G,G'}
    +
    V(q_\parallel,\bar k+G)
    \chi(q_\parallel,\omega;\bar k+G, \bar k+G')
    \rb
    \phi_{ext}(q_\parallel,\omega;\bar k+ G')
    \,\,\,,
\end{equation}
where I've collected the prefactor of \eqref{planar phi ind} into the Coulomb interaction,
\begin{equation}
    \label{V def}
    V(q_\parallel,\bar k+G):=
    \frac{e^2}{\e_0\lb q_\parallel^2+\lp\bar k+G\rp^2\rb}
    \,\,\,.
\end{equation}

The utility of the electrostatic screening response contained within \eqref{planar phi to phi ext relation} relies on noting that we can control the nature of the probe modeled by $\phi_{ext}$ and that we can decide what part of the total potential $\phi$ to calculate.  If we consider the long wavelength response to a long wavelength perturbation, or $G=G'=0$ in \eqref{planar phi to phi ext relation}, then we can reduce the matrix equation in \eqref{planar phi to phi ext relation} to the {\it scalar} relationship,
\begin{equation}
    \label{long wavelength linear response relationship}
    \phi(q_\parallel, Q, \omega) 
    =
    \lb
    1
    +
    V(q_\parallel, Q)
    \chi(q_\parallel, Q, \omega)
    \rb
    \phi_{ext}(q_\parallel, Q, \omega)
    \,\,\,.
\end{equation}