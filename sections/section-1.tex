\section{The electrostatic framework}
\label{section: electrostatic framework}

The electrostatic framework provides a direct means of calculating the electronic density response of a superlattice through an electrostatic screening analysis.  In particular, I will demonstrate how to obtain the frequency- and wavevector-resolved density response function in the long wavelength limit,\footnote{Here, the ``long wavelength limit'' refers to wavevectors along the superlattice direction that lie within the first Brillouin zone.}
\begin{equation}
    \label{Fourier chi}
    \chi(q,\omega):=\chi(q,q';\omega)_{q=q'}
    :=
    \lb\frac{1}{\vol}\int d^3 r d^3 r' e^{iq\cdot r} e^{-i q'\cdot r'}\chi(r,r';\omega)\rb_{q=q'}
    \,\,\,.
\end{equation}
By superlattice, I mean to consider a system with translation-invariance in the planar directions (those perpendicular to the superlattice dimension),\footnote{This assumption can be relaxed to the discrete translation symmetry of a crystalline solid, so long as the ``long wavelength limit'' is appropriately redefined.} and discrete translation symmetry in the layering direction.  By ``electrostatic,'' I mean to define the system under study in terms of electric potentials and charge densities, which restricts the calculation to the non-relativistic and non-retarded limit.  In terms of the wavevector $q$ and frequency $\omega$ of excitations within this framework, an electrostatic analysis roughly holds in the regime where $q^2\gg \omega^2/c^2$.


The electrostatic framework considers the response of an electronic, bound charge density to an externally-sourced perturbing potential, $\phi_{ext}$.  The perturbing field induces a screening charge density $\rho_{ind}$, which is governed by 
the electronic density response function $\chi$,\footnote{I.e., we are ignoring ionic contributions to the electrostatic screening response.  This neglected response would nevertheless contribute to charge density probes like EELS.}
\begin{equation}
    \label{real space chi def}
    \rho_{ind}(r,t) = e^2\int d^3 r' dt' \chi(r,r';t-t')\phi_{ext}(r',t')
    \,\,\,.
\end{equation}



As such, out-of-plane wavevectors are only conserved up to reciprocal lattice vectors,
\begin{equation}
    \label{Fourier chi expression}
    \chi(q_\parallel,\omega;z,z') = \sum_G e^{iG z} \int \frac{dk}{2\pi} e^{ik(z-z')}\chi(q_\parallel,\omega;k+G,k)
    \,\,\,,
\end{equation}
where $q_\parallel$ is the planar component of the Fourier wavevector and the sum is over all out-of-plane reciprocal vectors.  As is standard, the continuous out-of-plane wavevector $k$ in \eqref{Fourier chi expression} can be restricted to the first Brillouin zone (denoted as $\bar k$ in what follows),
\begin{equation}
    \label{Fourier chi G matrix}
    \chi(q_\parallel,\omega;z,z') = \sum_{G,G'} e^{iG z} e^{-i G' z'} \int_\text{BZ} \frac{d\bar k}{2\pi} e^{i\bar k(z-z')}\chi(q_\parallel,\omega;\bar k+G,\bar k+G')
    \,\,\,.
\end{equation}





The linear response relationship in \eqref{real space chi def} can be expressed in terms of the discrete superlattice translation symmetry in \eqref{Fourier chi G matrix} as a matrix equation in reciprocal lattice vectors,
\begin{equation}
    \label{planar linear response}
     \rho_{ind}(q_\parallel,\omega;\bar k+G) = e^2\sum_{G'}\chi(q_\parallel,\omega;\bar k+G, \bar k+G')
     \phi_{ext}(q_\parallel,\omega;\bar k+G')
     \,\,\,.
\end{equation}


The screening field $\phi_{ind}$ is sourced by an induced charge density $\rho_{ind}$ through Gauss's law; in Fourier space,
\begin{equation}
    \label{Fourier phi ind}
    \phi_{ind}(q,\omega) = \frac{\rho_{ind}(q,\omega)}{\e_0 q^2}
    \,\,\,.
\end{equation}


An analogous matrix equation results for the screening field,
\begin{equation}
    \label{planar phi ind}
     \phi_{ind}(q_\parallel,\omega;\bar k+G) = \frac{e^2}{\e_0\lb q_\parallel^2+\lp\bar k+G\rp^2\rb} \sum_{G'}\chi(q_\parallel,\omega;\bar k+G, \bar k+G')
     \phi_{ext}(q_\parallel,\omega;\bar k+G')
     \,\,\,.
\end{equation}


At the level of linear response, the (total) potential $\phi$ results from the sum
\begin{equation}
    \label{phi def}
    \phi(q,\omega) = \phi_{ind}(q,\omega) + \phi_{ext}(q,\omega)
    \,\,\,.
\end{equation}


The expression for the (total) potential $\phi$ can similarly be written in terms of $\phi_{ext}$ through its definition in \eqref{phi def} as
\begin{equation}
    \label{planar phi to phi ext relation}
    \phi(q_\parallel,\omega;\bar k+G) 
    =
    \sum_{G'}
    \lb
    \delta_{G,G'}
    +
    V(q_\parallel,\bar k+G)
    \chi(q_\parallel,\omega;\bar k+G, \bar k+G')
    \rb
    \phi_{ext}(q_\parallel,\omega;\bar k+ G')
    \,\,\,,
\end{equation}
where we've collected the prefactor of \eqref{planar phi ind} into the Coulomb interaction,
\begin{equation}
    \label{V def}
    V(q_\parallel,\bar k+G):=
    \frac{e^2}{\e_0\lb q_\parallel^2+\lp\bar k+G\rp^2\rb}
    \,\,\,.
\end{equation}